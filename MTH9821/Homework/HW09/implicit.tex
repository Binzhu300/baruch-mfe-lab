%%%%%%%%%%%%%%%%%%%%%%%%%%%%%%%%%%%%%%%%%
% Short Sectioned Assignment
% LaTeX Template
% Version 1.0 (5/5/12)
%
% This template has been downloaded from:
% http://www.LaTeXTemplates.com
%
% Original author:
% Frits Wenneker (http://www.howtotex.com)
%
% License:
% CC BY-NC-SA 3.0 (http://creativecommons.org/licenses/by-nc-sa/3.0/)
%
%%%%%%%%%%%%%%%%%%%%%%%%%%%%%%%%%%%%%%%%%

%----------------------------------------------------------------------------------------
%	PACKAGES AND OTHER DOCUMENT CONFIGURATIONS
%----------------------------------------------------------------------------------------

\documentclass[paper=a4, fontsize=10pt,]{scrartcl} % A4 paper and 11pt font size


\usepackage{geometry}
 \geometry{
 a4paper,
 total={170mm,257mm},
 left=30mm,
 right=30mm,
 top=20mm,
 }



\usepackage[T1]{fontenc} % Use 8-bit encoding that has 256 glyphs
%\usepackage{fourier} % Use the Adobe Utopia font for the document - comment this line to return to the LaTeX default
\usepackage[english]{babel} % English language/hyphenation
\usepackage{amsmath,amsfonts,amsthm, amssymb} % Math packages
\usepackage{enumitem}
\usepackage{cancel}
\usepackage{graphicx} % Required to insert images

\newtheorem{theorem}{Theorem}[section]
\newtheorem{corollary}{Corollary}[theorem]
\newtheorem{lemma}[theorem]{Lemma}
\theoremstyle{theorem}
\newtheorem{definition}{Definition}[section]
\theoremstyle{remark}
\newtheorem*{remark}{Remark}
\theoremstyle{example}
\newtheorem{example}{Example}[section]
 
\usepackage{hhline}
\usepackage{sectsty} % Allows customizing section commands
\allsectionsfont{\centering \normalfont\normalsize\scshape} % Make all sections centered, the default font and small caps

\usepackage{fancyhdr} % Custom headers and footers
\pagestyle{fancyplain} % Makes all pages in the document conform to the custom headers and footers
\fancyhead{} % No page header - if you want one, create it in the same way as the footers below
\fancyfoot[L]{} % Empty left footer
\fancyfoot[C]{} % Empty center footer
\fancyfoot[R]{\thepage} % Page numbering for right footer
\renewcommand{\headrulewidth}{0pt} % Remove header underlines
\renewcommand{\footrulewidth}{0pt} % Remove footer underlines
\setlength{\headheight}{13.6pt} % Customize the height of the header

\numberwithin{equation}{section} % Number equations within sections (i.e. 1.1, 1.2, 2.1, 2.2 instead of 1, 2, 3, 4)
\numberwithin{figure}{section} % Number figures within sections (i.e. 1.1, 1.2, 2.1, 2.2 instead of 1, 2, 3, 4)
\numberwithin{table}{section} % Number tables within sections (i.e. 1.1, 1.2, 2.1, 2.2 instead of 1, 2, 3, 4)

\setlength\parindent{0pt} % Removes all indentation from paragraphs - comment this line for an assignment with lots of text




\begin{document}

	
\begin{center}
\scshape\fontseries{bx}\selectfont \textsc{\textbf{ MTH9821. Implicit Finite Difference Method for American Options Based on Projected $LU$-Decomposition }}
\end{center}

\begin{center}
\footnotesize{ShengQuan Zhou}\\
\today
\end{center}


\begin{abstract}
   \textsc{Abstract}. This note outlines an implicit finite difference method based on $LU$-decomposition and projected backward substitution
   for American options.
   Unlike the projected SOR method which iteratively improves the solution to a given tolerance, this algorithm
    takes a finite number of operations to complete and yields full arithmetic precision. 
   Numerical tests on an example of American option in Homework Assignment \#9 are given at the end.
\end{abstract}


\section{Matrix Inequalities for Backward Euler Method}
The backward Euler method for American options involves the solution of the following matrix inequalities at each time step:
\begin{align}\label{ineq1}
\mathbf{A}\cdot\mathbf{u} &\ge \mathbf{b}, \\
\label{ineq11}\mathbf{u} &\ge \mathbf{g},
\end{align}
while entry-wise, one of two inequalities is an equality. Here, 
$$
\mathbf{A} = \begin{pmatrix}
1+2\alpha & -\alpha & 0 & \cdots & 0\\
-\alpha & 1+2\alpha & -\alpha &  & 0\\
0 & -\alpha &\ddots & \ddots & 0\\
\vdots & & \ddots & 1+2\alpha & -\alpha\\
0&\cdots & 0 &-\alpha & 1+2\alpha
\end{pmatrix},
$$
the vector $\mathbf{b}$ is given by a combination of the state vector at the previous time step and the boundary conditions, and
the vector $\mathbf{g}$ is the early exercise premium.\\

The $LU$-decomposition of the matrix $\mathbf{A}$ can be written as
$$
\begin{pmatrix}
1+2\alpha & -\alpha & 0 & \cdots & 0\\
-\alpha & 1+2\alpha & -\alpha &  & 0\\
0 & -\alpha &\ddots & \ddots & 0\\
\vdots & & \ddots & 1+2\alpha & -\alpha\\
0&\cdots & 0 &-\alpha & 1+2\alpha
\end{pmatrix}
= \underbrace{\begin{pmatrix}
1 & 0 & 0 & \cdots & 0\\
-\frac{\alpha}{y_0} & 1 & \ddots &  & 0\\
0 & \ddots &\ddots & \ddots & 0\\
\vdots & & \ddots & \ddots & 0\\
0&\cdots & 0 &-\frac{\alpha}{y_{N-1}} & 1
\end{pmatrix}}_{\triangleq \mathbf{L}}\cdot
\underbrace{\begin{pmatrix}
y_0 & -\alpha & 0 & \cdots & 0\\
0 & y_1 & \ddots &  & 0\\
0 & \ddots &\ddots & \ddots & 0\\
\vdots & & \ddots & \ddots & -\alpha\\
0&\cdots & 0 &0 & y_{N}
\end{pmatrix}}_{\triangleq \mathbf{U}},
$$
where
\begin{align}\label{recursion}
y_0 &= 1+2\alpha,\\
y_n &= (1+2\alpha) -\frac{\alpha^2}{y_{n-1}}, \quad n=1,\cdots,N;
\end{align}

\section{Transformation to Triangular Matrix Inequalities}

\begin{lemma}
The matrix inequality \ref{ineq1} is equivalent to the upper-triangular matrix inequalities:
\begin{align}\label{ineq2}
\mathbf{v} &\triangleq \mathbf{U}\cdot\mathbf{u} \ge \mathbf{L}^{-1}\cdot\mathbf{b}
\end{align}
if $\alpha$ is sufficiently small.
\end{lemma}
\begin{proof}
The above statement amounts to saying that $y_n>0$, $n=0,\cdots,N$, so that the following sequence
of inequalities hold true:
\begin{align*}
v_0 &\ge b_0,\\
v_1 &\ge b_1 + \frac{\alpha}{y_0}v_0 \ge b_1 + \frac{\alpha}{y_0}b_0,\\
&\cdots \\
v_N &\ge b_N + \frac{\alpha}{y_{N-1}}v_{N-1} \ge \cdots \ge b_N + \frac{\alpha}{y_{N-1}}\Big(b_{N-1} + \frac{\alpha}{y_{N-2}}\Big( 
b_{N-2} + \frac{\alpha}{y_{N-3}}\Big(b_{N-3}+\cdots + \frac{\alpha}{y_{0}}b_0 \Big)\Big)\Big).
\end{align*}
Define $Y_n = \Pi_{i=0}^n y_n$, we get a $2^{\text{nd}}$ order linear recurrence relation $Y_n - (1+2\alpha)Y_{n-1} +\alpha^2 Y_{n-2}=0$. 
The fact that $y_n>0$, $n=0,\cdots,N$ for sufficiently small $\alpha$ can be proved by solving the above recurrence relation 
explicitly and let $y_n = \frac{Y_n}{Y_{n-1}}$. Detail is omitted.
\end{proof}

\begin{lemma}
If the matrix inequiality \ref{ineq1} is an equality for an entry $n$, i.e. $(\mathbf{A}\cdot\mathbf{u} - \mathbf{b})_n=0$, then the
triangular matrix inequality \ref{ineq2} is also an equality for the same entry, i.e. $(\mathbf{U}\cdot\mathbf{u} - \mathbf{L}^{-1}\cdot\mathbf{b})_n
=0$.
\end{lemma}
\begin{proof}
The proof is based on the existence of an early exercise boundary entry $\hat{n}$ such that
\begin{align*}
&(\mathbf{A}\cdot\mathbf{u} - \mathbf{b})_n=0, \quad (\mathbf{u} - \mathbf{g})_n>0, \quad \text{for } 0\le n < \hat{n};\\
&(\mathbf{A}\cdot\mathbf{u} - \mathbf{b})_n>0, \quad (\mathbf{u} - \mathbf{g})_n=0, \quad \text{for } \hat{n}\le n\le N,
\end{align*}
which is heuristically evident but by itself requires a proof. Detail is omitted. Note that we adopted an indexing convention that in the finite difference
scheme for an American option, the right boundary of the computational domain is labeled by $n=0$ and the left boundary by $n=N$.
If the indexing convention is the other way around, we need a $UL$-decomposition instead.\\

Now, for a given $n$, if  $(\mathbf{A}\cdot\mathbf{u} - \mathbf{b})_n=0$, then
$$
(\mathbf{A}\cdot\mathbf{u} - \mathbf{b})_{n'}=0, \quad \forall n'<n.
$$
thus, invoking the lower-triangularity of $\mathbf{L}^{-1}$, we have
\begin{align*}
(\mathbf{U}\cdot\mathbf{u} - \mathbf{L}^{-1}\cdot\mathbf{b})_n &= \left[\mathbf{L}^{-1} \cdot(\mathbf{A}\cdot \mathbf{u} - \mathbf{b})\right]_n\\
&= \sum_{n'\le n}\mathbf{L}^{-1}_{nn'} \cdot(\mathbf{A}\cdot \mathbf{u} - \mathbf{b})_{n'}\\
&= 0.
\end{align*}
\end{proof}
The above result establishes the equivalence between the matrix inequalities \ref{ineq1} - \ref{ineq11} and
\begin{align*}
\mathbf{U}\cdot\mathbf{u} &\ge \mathbf{L}^{-1}\cdot \mathbf{b}, \\
\mathbf{u} &\ge \mathbf{g},
\end{align*}
while entry-wise, one of two inequalities is an equality.

\begin{lemma}
Denote $\mathbf{c} \triangleq \mathbf{L}^{-1}\cdot\mathbf{b}$, the triangular matrix inequalities
\begin{align}\label{ineq3}
\mathbf{U}\cdot\mathbf{u} &\ge \mathbf{c}, \\
\label{ineq4} \mathbf{u} &\ge \mathbf{g},
\end{align}
while entry-wise, one of two inequalities is an equality, can be solved by the following projected backward substitution procedure:
\begin{align*}
u_{N} &= \max\left\{ \frac{c_N}{y_N}, g_N \right\},\\
u_{n} &=  \max\left\{ \frac{c_n + \alpha u_{n+1}}{y_n}, g_n\right\}, \quad n=N-1,\cdots, 0.
\end{align*}
\end{lemma}
\begin{proof}
The proof is essentially an induction based on the triangularity of the system and the positiveness of $y_n$, $0\le n\le N$. Detail is omitted.
\end{proof}
%The equivalence between the matrix inequalities \ref{ineq1}-\ref{ineq11} and \ref{ineq3}-\ref{ineq4} is established.

\section{Numerical Tests}
Numerical tests are performed on the example of American option valuation in Homework \#9, focusing on the comparison between backward Euler method by projected $LU$ and by projected SOR. Similar tests are done for Crank-Nicolson. 
\begin{footnotesize}
\begin{center}
\begin{tabular}{l*{6}{c}r}
Method              &  Error I &  Error II & $\Delta$ & $\Gamma$ & $\Theta$  & Error III \\
\hhline{=======}
Backward Euler by $LU$  & 0.003959021	& 0.004001460	& -0.379403240	& 0.030962375	& -2.767599900	& 0.001595845 \\
Backward Euler by SOR   & 0.003959004	& 0.004001443	& -0.379403240	& 0.030962375	& -2.767599886	& 0.001595643 \\
\hline
Crank-Nicolson by $LU$  & 0.001037275	& 0.001079720	& -0.379476930	& 0.030920454	& -2.765847325	& 0.000887689 \\
Crank-Nicolson by SOR   & 0.001037275	& 0.001079720	& -0.379476930	& 0.030920453	& -2.765847326	& 0.000887694 \\
\hline
\end{tabular}
\end{center}
Note: The domain in discretized by $M=256$ and $\alpha_{\text{temp}}=0.45$. In SOR method, $\omega=1.2$ and $tol=10^{-6}$.
\end{footnotesize}



\end{document}