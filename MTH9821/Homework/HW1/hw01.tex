%%%%%%%%%%%%%%%%%%%%%%%%%%%%%%%%%%%%%%%%%
% Short Sectioned Assignment
% LaTeX Template
% Version 1.0 (5/5/12)
%
% This template has been downloaded from:
% http://www.LaTeXTemplates.com
%
% Original author:
% Frits Wenneker (http://www.howtotex.com)
%
% License:
% CC BY-NC-SA 3.0 (http://creativecommons.org/licenses/by-nc-sa/3.0/)
%
%%%%%%%%%%%%%%%%%%%%%%%%%%%%%%%%%%%%%%%%%

%----------------------------------------------------------------------------------------
%	PACKAGES AND OTHER DOCUMENT CONFIGURATIONS
%----------------------------------------------------------------------------------------

\documentclass[paper=a4, fontsize=11pt]{scrartcl} % A4 paper and 11pt font size

\usepackage[T1]{fontenc} % Use 8-bit encoding that has 256 glyphs
\usepackage{fourier} % Use the Adobe Utopia font for the document - comment this line to return to the LaTeX default
\usepackage[english]{babel} % English language/hyphenation
\usepackage{amsmath,amsfonts,amsthm, amssymb} % Math packages
\usepackage{enumitem}
\usepackage{cancel}
\usepackage{graphicx} % Required to insert images

\usepackage{lipsum} % Used for inserting dummy 'Lorem ipsum' text into the template

\usepackage{sectsty} % Allows customizing section commands
\allsectionsfont{\centering \normalfont\scshape} % Make all sections centered, the default font and small caps

\usepackage{fancyhdr} % Custom headers and footers
\pagestyle{fancyplain} % Makes all pages in the document conform to the custom headers and footers
\fancyhead{} % No page header - if you want one, create it in the same way as the footers below
\fancyfoot[L]{} % Empty left footer
\fancyfoot[C]{} % Empty center footer
\fancyfoot[R]{\thepage} % Page numbering for right footer
\renewcommand{\headrulewidth}{0pt} % Remove header underlines
\renewcommand{\footrulewidth}{0pt} % Remove footer underlines
\setlength{\headheight}{13.6pt} % Customize the height of the header

\numberwithin{equation}{section} % Number equations within sections (i.e. 1.1, 1.2, 2.1, 2.2 instead of 1, 2, 3, 4)
\numberwithin{figure}{section} % Number figures within sections (i.e. 1.1, 1.2, 2.1, 2.2 instead of 1, 2, 3, 4)
\numberwithin{table}{section} % Number tables within sections (i.e. 1.1, 1.2, 2.1, 2.2 instead of 1, 2, 3, 4)

\setlength\parindent{0pt} % Removes all indentation from paragraphs - comment this line for an assignment with lots of text
\usepackage{color}
\definecolor{dkgreen}{rgb}{0,0.6,0}
\definecolor{gray}{rgb}{0.5,0.5,0.5}
\definecolor{mauve}{rgb}{0.58,0,0.82}
\usepackage{listings}
\lstset{frame=tb,
  language=C++,
  aboveskip=3mm,
  belowskip=3mm,
  showstringspaces=false,
  columns=flexible,
  basicstyle={\small\ttfamily},
  numbers=none,
  numberstyle=\tiny\color{gray},
  keywordstyle=\color{blue},
  commentstyle=\color{dkgreen},
  stringstyle=\color{mauve},
  breaklines=true,
  breakatwhitespace=true
  tabsize=3
}



%----------------------------------------------------------------------------------------
%	TITLE SECTION
%----------------------------------------------------------------------------------------

\newcommand{\horrule}[1]{\rule{\linewidth}{#1}} % Create horizontal rule command with 1 argument of height

\title{	
\normalfont \normalsize
\textsc{Baruch, MFE} \\ [25pt] % Your university, school and/or department name(s)
\horrule{0.5pt} \\[0.4cm] % Thin top horizontal rule
\huge MTH 9821 Homework One \footnote{Team work:\newline
Chu, Hongshan \#1, \#2, \#3 \newline 
Wang, Jiaxi \#4, \#5 \newline
Wei, Zhaoyue \#5, \#6 \newline 
Yin, Gongshun \#7, \#8 \newline
Zhou, ShengQuan \#1, \#2, \#3, \#7 ,\#8}\\  % The assignment title
\horrule{2pt} \\[0.5cm] % Thick bottom horizontal rule
}


\author{Chu, Hongshan\\
Wang, Jiaxi\\
Wei, Zhaoyue\\
Yin, Gongshun\\
Zhou, ShengQuan} % Your name

\date{\normalsize\today} % Today's date or a custom date

\begin{document}
	


\maketitle % Print the title

\newpage



\section{Uniqueness of $LU$-Decomposition}
Let $L_1$ and $L_2$ be nonsingular lower triangular matrices and let $U_1$ and $U_2$ be nonsingular upper triangular
matrices. If $L_1 U_1 = L_2 U_2$, show that there exists a nonsingular diagonal matrix $D$ such that
$$
L_1 = L_2 D, \quad \text{and} \quad U_1 = D^{-1}U_2.
$$\\
\textit{Proof}: Since $L_1$, $L_2$, $U_1$, and $U_2$ are nonsingular, apply $L_2^{-1}$ from the left and $U_1^{-1}$ from the right:
\begin{eqnarray}
\nonumber L_1 U_1 = L_2 U_2 &\Rightarrow& L_2^{-1} L_1 U_1 U_1^{-1} = L_2^{-1}L_2 U_2 U_1^{-1}\\
\nonumber &\Rightarrow & \underbrace{L_2^{-1} L_1}_{\text{lower-triangular}} = \underbrace{U_2 U_1^{-1}}_{\text{upper-trianglar}} \\
\nonumber &\Rightarrow& L_2^{-1} L_1 = U_2 U_1^{-1} = D,
\end{eqnarray}
where $D$ is a nonsingular diagonal matrix. In other words,
$$
L_1 = L_2 D, \quad \text{and} \quad U_1 = D^{-1}U_2.
$$


\newpage


\section{$LU$-Decomposition of a Special Matrix}
Find the $LU$-decomposition without pivoting of the matrix
$$
\begin{pmatrix}
1 & 0 & 0 & 0 & 1 \\
-1 & 1 & 0 & 0 & 1 \\
-1 & -1 & 1 & 0 & 1 \\
-1 & -1 & -1 & 1 & 1 \\
-1 & -1 & -1 & -1 & 1 \\
\end{pmatrix}
$$\\
\textit{Solution}: This $LU$-decomposition can be done by hand exactly:
$$
\begin{pmatrix}
1 & 0 & 0 & 0 & 1 \\
-1 & 1 & 0 & 0 & 1 \\
-1 & -1 & 1 & 0 & 1 \\
-1 & -1 & -1 & 1 & 1 \\
-1 & -1 & -1 & -1 & 1 \\
\end{pmatrix}
= \begin{pmatrix}
1 & 0 & 0 & 0 & 0 \\
-1 & 1 & 0 & 0 & 0 \\
-1 & -1 & 1 & 0 & 0 \\
-1 & -1 & -1 & 1 & 0 \\
-1 & -1 & -1 & -1 & 1 \\
\end{pmatrix}
\times
\begin{pmatrix}
1 & 0 & 0 & 0 & 1 \\
0 & 1 & 0 & 0 & 2 \\
0 & 0 & 1 & 0 & 4 \\
0 & 0 & 0 & 1 & 8 \\
0 & 0 & 0 & 0 & 16 \\
\end{pmatrix}.
$$
The entries of $U$ on the rightmost column are unbound relative to the those of the original matrix as the matrix dimension grows. This is a classic example of a matrix whole $LU$-decomposition is unstable.

\newpage

\section{$LU$-Decomposition with No-Pivoting and Row-Pivoting}
Let
$$
A = \begin{pmatrix}
2 & -1 & 1 \\
-2 & 1 & 3 \\
4 & 0 & -1 \\
\end{pmatrix}
$$\\
(i) Show that the $2\times 2$ leading principal minor of $A$ is 0.\\
\textit{Proof}: Direct computation,
$$
\det\begin{pmatrix}
2 & -1 \\
-2 & 1 \\
\end{pmatrix} =  2\times 1- (-1)\times(-2) = 2-2 = 0.
$$\\

(ii) Attempt to do the $LU$-decomposition without pivoting of the matrix $A$, and show that the division by $U_{22}$ cannot be performed when trying to compute
the second row of $L$.\\
\textit{Solution}: We perform $LU$-decomposition with no pivoting at stages:\\
\newline
Stage 1:
$$
A = \begin{pmatrix}
2 & -1 & 1 \\
-2 & 1 & 3 \\
4 & 0 & -1 \\
\end{pmatrix}
=
\begin{pmatrix}
1 & 0 & 0 \\
-1 & \times & 0 \\
2 & \times & \times \\
\end{pmatrix}
\times
\begin{pmatrix}
2 & -1 & 1 \\
0 & \times & \times \\
0 & 0 & \times \\
\end{pmatrix}
$$

Stage 2: we have
$$
\begin{pmatrix}
-1  \\
2 \\
\end{pmatrix}
\times
\begin{pmatrix}
-1 & 1 \\
\end{pmatrix}
=
\begin{pmatrix}
1 & -1\\
-2 & 2 \\
\end{pmatrix}
\Rightarrow
A = \begin{pmatrix}
2 & -1 & 1 \\
-2 & 1 & 3 \\
4 & 0 & -1 \\
\end{pmatrix}
\rightarrow
\begin{pmatrix}
2 & -1 & 1 \\
-2 & \boxed{0} & 4 \\
4 & 2 & -3 \\
\end{pmatrix}
$$
At this stage, we get $U_{22} = A_{22}=0$. Thus, the division by $U_{22}$ cannot be performed when trying to compute
the second row of $L$.\\

(iii) Show that the matrix $A$ is nonsingular, and compute the $LU$-decomposition with row pivoting of $A$.\\
\textit{Solution}: Let $P = (1,2,3)$. We perform $LU$-decomposition with row pivoting at stages:\\
\newline
Stage 1: Interchange row 1 and row 3, $P = (3,2,1)$,
$$
A \rightarrow \begin{pmatrix}
4 & 0 & -1 \\
-2 & 1 & 3 \\
2 & -1 & 1 \\
\end{pmatrix}
=
\begin{pmatrix}
1 & 0 & 0 \\
-1/2 & \times & 0 \\
1/2 & \times & \times \\
\end{pmatrix}
\times
\begin{pmatrix}
4 & 0 & -1 \\
0 & \times & \times \\
0 & 0 & \times \\
\end{pmatrix}
$$
Stage 2:
$$
\begin{pmatrix}
-1/2  \\
1/2 \\
\end{pmatrix}
\times
\begin{pmatrix}
0 & -1 \\
\end{pmatrix}
=
\begin{pmatrix}
0 & 1/2\\
0 & -1/2 \\
\end{pmatrix}
\Rightarrow
A \rightarrow
\begin{pmatrix}
4 & 0 & -1 \\
-2 & 1 & 5/2 \\
2 & -1 & 3/2 \\
\end{pmatrix}
$$
No exchange of rows is needed,
$$
A \rightarrow
\begin{pmatrix}
4 & 0 & -1 \\
-2 & 1 & 5/2 \\
2 & -1 & 3/2 \\
\end{pmatrix}
=
\begin{pmatrix}
1 & 0 & 0 \\
-1/2 & 1 & 0 \\
1/2 & -1  & \times \\
\end{pmatrix}
\times
\begin{pmatrix}
4 & 0 & -1 \\
0 & 1 & 5/2 \\
0 & 0 & \times \\
\end{pmatrix}.
$$
Stage 3: $-1\times 5/2 = -5/2$. Thus, no exchange of rows,
$$
A \rightarrow
\begin{pmatrix}
4 & 0 & -1 \\
-2 & 1 & 5/2 \\
2 & -1 & 4 \\
\end{pmatrix}
=
\begin{pmatrix}
1 & 0 & 0 \\
-1/2 & 1 & 0 \\
1/2 & -1  & 1 \\
\end{pmatrix}
\times
\begin{pmatrix}
4 & 0 & -1 \\
0 & 1 & 5/2 \\
0 & 0 & 4 \\
\end{pmatrix}.
$$
In summary,
$$
\underbrace{\begin{pmatrix}
0 & 0 & 1 \\
0 & 1 & 0 \\
1 & 0  & 0 \\
\end{pmatrix}}_{P}
\times
\underbrace{\begin{pmatrix}
2 & -1 & 1 \\
-2 & 1 & 3 \\
4 & 0 & -1 \\
\end{pmatrix}}_{A}
=
\underbrace{\begin{pmatrix}
1 & 0 & 0 \\
-1/2 & 1 & 0 \\
1/2 & -1  & 1 \\
\end{pmatrix}}_{L}
\times
\underbrace{\begin{pmatrix}
4 & 0 & -1 \\
0 & 1 & 5/2 \\
0 & 0 & 4 \\
\end{pmatrix}}_{U}.
$$
The existence of an $LU$-decomposition with row pivoting serves as a proof that the matrix $A$ is nonsingular.

\newpage
\section{Pseudocode for the Forward Substitution for a Lower Triangular Banded matrix}
Write the pseudocode for the forward substitution corresponding to a lower triangular banded matrix of band $m$, i.e. for solving
$Lx=b$ where $b$ is an $n\time 1$ vector and $L$ is an $n\times n$ lower triangular matrix such that
$$
L_{jk} = 0, \quad \forall 1\le j,k\le n, \quad j-k>m.
$$
What is the corresponding operation count?
\begin{lstlisting}
Function Call:
x = forward_subst_banded(L,b)


Input:
L = nonsingular lower triangular banded matrix of band m of size n
b = column vector of size n

Output:
x = solution to Lx=b

x(1) = b(1)/L(1,1);

for j = 2 : n
    sum = 0;
    for k = max{1,j-m} : (j-1)
        sum = sum + L(j,k)x(k);
    end
    x(j) = (b(j) - sum)/L(j,j);
end
\end{lstlisting}
The operation count of the above forward substitution for a lower triangular banded matrix is given by
$$
1 + \sum_{j=2}^{m+1} \left[2(j-1)+2\right] + \sum_{j=m+2}^{n} \left(2m+2\right) = 2mn - m^2 + O(m).
$$

\newpage
\section{Pseudocode for the Backward Substitution for an Upper Triangular Banded matrix}
Write the pseudocode for the backward substitution corresponding to an upper triangular banded matrix of band $m$, i.e. for solving
$Ux=b$ where $b$ is an $n\time 1$ vector and $U$ is an $n\times n$ upper triangular matrix such that
$$
U_{jk} = 0, \quad \forall 1\le j,k\le n, \quad k-j>m.
$$
What is the corresponding operation count?
\begin{lstlisting}
Function Call:
x = backward_subst_banded(U,b)


Input:
U = nonsingular upper triangular banded matrix of band m of size n
b = column vector of size n

Output:
x = solution to Ux=b

x(n) = b(n)/U(n,n);

for j = (n-1) : 1
    sum = 0;
    for k = (j+1) : min{n,j+m}
        sum = sum + U(j,k)x(k);
    end
    x(j) = (b(j) - sum)/U(j,j);
end
\end{lstlisting}
The operation count of the above forward substitution for a lower triangular banded matrix is given by
$$
1 + \sum_{j=n-m}^{n-1} \left[2(n-j)+2\right] + \sum_{j=1}^{n-m-1} \left(2m+2\right) = 2mn - m^2 + O(m).
$$

\newpage
\newpage
\section{Pseudocode for the $LU$-Decomposition without Pivoting for Banded Matrices}
\subsection{$LU$-Decomposition without Pivoting}
Write the pseudocode for the $LU$-decomposition without pivoting for banded matrices of band $m$. What is the operation count?
\begin{lstlisting}
Function Call:
[L,U]=lu_no_pivoting_banded(A)

Input:
A = nonsingular banded matrix of band m

Output:
L = lower triangular matrix with entries 1 on main diagonal
U = upper triangular matrix
such that A = LU

for i = 1:(n-1)
    for k = i:n
        U(i,k) = A(i,k);
        L(k,i) = A(k,i)/U(i,i);
    end
    for j = (i+1):n
        for k = (i+1):n
            A(j,k) = A(j,k) - L(j,i)U(i,k);
        end
    end
end

L(n,n)=1; U(n,n)=A(n,n);
\end{lstlisting}

The derivation of the operation count is given in the lecture,
$$
\text{Operation Count} = \frac{2}{3}n^3+O\left(n^2\right).
$$

\newpage
\subsection{$LU$-Decomposition without Pivoting: Band Simplification}
Use (without proving) the fact that the $L$ and $U$ factors from the $LU$-decompostion without pivoting of a banded
matrix of band $m$ are a banded lower triangular matrix of band $m$ and a banded upper triangular matrix of band $m$, respectively.
What is the corresponding operation count?
\begin{lstlisting}

Function Call:
[L,U]=lu_no_pivoting_banded(A)

Input:
A = nonsingular banded matrix of band m

Output:
L = lower triangular matrix with entries 1 on main diagonal
U = upper triangular matrix
such that A = LU

L=0, U=0; // all entries zero
for i = 1:(n-1)
    for k = i : min{n,i+m}
        U(i,k) = A(i,k);
        L(k,i) = A(k,i)/U(i,i);
    end
    
    for j = (i+1) : min{n,i+m}
        for k = (i+1) : min{n,i+m}
            A(j,k) = A(j,k) - L(j,i)U(i,k);
        end
    end
end

L(n,n)=1; U(n,n)=A(n,n);
\end{lstlisting}
The operation count is divided into two parts at $i=n-m$:
$$
\sum_{i=1}^{n-m-1}\left( m+1 + \underbrace{\sum_{j=1}^{i+m}\sum_{j=1}^{i+m}2}_{=2m^2} \right) + \sum_{i=n-m}^{n-1}\left( n-i+1+ \underbrace{\sum_{j=1}^{n-i}\sum_{j=1}^{n-i}2}_{2(n-i)^2} \right) = \left(2m^2 + m +1 \right) n - \frac{4}{3}m^3 + O\left(m^2\right).
$$

\newpage
\section{C++ Codes for Backward and Forward Substitution}

\begin{lstlisting}
#include <triangular_solve.h>
#include <Eigen/Dense>
#include <cassert>

Eigen::VectorXd forward_subst(const Eigen::MatrixXd & L,
                                 const Eigen::VectorXd & b)
{
    int n = b.size();
    assert(L.rows() == n);
    assert(L.cols() == n);

    Eigen::VectorXd x(n);
    x(0) = b(0)/L(0,0);
    for (int i=1; i<n; i++) {
        double sum = 0;
        for (int j=0; j<i; j++) {
            sum += L(i,j)*x(j);
        }
        x(i) = (b(i)-sum)/L(i,i);
    }

    return x;
}

Eigen::VectorXd backward_subst(const Eigen::MatrixXd & U,
                                  const Eigen::VectorXd & b)
{
    int n = b.size();
    assert(U.rows() == n);
    assert(U.cols() == n);

    Eigen::VectorXd x(n);
    x(n-1) = b(n-1)/U(n-1,n-1);
    for (int i=n-2; i>=0; i--) {
        double sum = 0;
        for (int j=i+1; j<n; j++) {
            sum += U(i,j)*x(j);
        }
        x(i) = (b(i)-sum)/U(i,i);
    }

    return x;
}
\end{lstlisting}

\newpage
\section{C++ Codes for $LU$-Decomposition}
\begin{lstlisting}
#include <lu.h>
#include <Eigen/Dense>
#include <cassert>
#include <tuple>

static void lu_helper(int k, int n,
                      Eigen::MatrixXd * A,
                      Eigen::MatrixXd * L,
                      Eigen::MatrixXd * U)
{
    for (int i=k; i<n; i++) {
        (*U)(k,i) = (*A)(k,i);
        (*L)(i,k) = (*A)(i,k)/(*U)(k,k);
    }

    for (int i=k+1; i<n; i++) {
        for (int j=k+1; j<n; j++) {
            (*A)(i,j) -= (*L)(i,k)*(*U)(k,j);
        }
    }
}

std::tuple<Eigen::MatrixXd, Eigen::MatrixXd>
    lu_no_pivoting(const Eigen::MatrixXd & A)
{
    Eigen::MatrixXd Acopy = A;
    int n = Acopy.rows();
    assert(n == Acopy.cols());

    Eigen::MatrixXd L(n,n);
    Eigen::MatrixXd U(n,n);

    L.triangularView<Eigen::StrictlyUpper>().setZero();
    U.triangularView<Eigen::StrictlyLower>().setZero();

    for (int k=0; k<n-1; k++) {
        lu_helper(k, n, &Acopy, &L, &U);
    }

    L(n-1,n-1) = 1;
    U(n-1,n-1) = Acopy(n-1,n-1);

    return std::make_tuple(L,U);
}



std::tuple<Eigen::VectorXi, Eigen::MatrixXd, Eigen::MatrixXd>
    lu_row_pivoting(const Eigen::MatrixXd & A)
{
    Eigen::MatrixXd Acopy = A;
    int n = Acopy.rows();
    assert(n == Acopy.cols());

    Eigen::VectorXi p = Eigen::VectorXi::LinSpaced(n,1,n);
    Eigen::MatrixXd L = Eigen::MatrixXd::Zero(n,n);;
    Eigen::MatrixXd U = Eigen::MatrixXd::Zero(n,n);;

    for (int k=0; k<n-1; k++) {
        int maxRow, maxCol;
        Acopy.block(k,k,n-k,1).array().abs().maxCoeff(&maxRow, &maxCol);
        Acopy.row(k).swap(Acopy.row(maxRow+k));
        p.row(k).swap(p.row(maxRow+k));
        L.row(k).swap(L.row(maxRow+k));
        lu_helper(k, n, &Acopy, &L, &U);
    }

    L(n-1,n-1) = 1;
    U(n-1,n-1) = Acopy(n-1,n-1);

    return std::make_tuple(p,L,U);
}
\end{lstlisting}

\end{document}
